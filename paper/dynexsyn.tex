\documentclass[a4paper]{llncs}
\usepackage{alltt,xspace,listings,boxedminipage,url,cite}
\def\pico{\textsc{Pico}\xspace}
\def\asfsdf{\textsc{Asf+Sdf}\xspace}
\def\cherrylisp{\textsc{CherryLisp}\xspace}

\def\mytextbf#1{\textrm{\textbf{#1}}}
\def\mytextit#1{\textrm{\textit{#1}}}
\def\Begin{\mytextbf{begin}}
\def\Declare{\mytextbf{declare}}
\def\End{\mytextbf{end}}
\def\Skip{\mytextbf{skip}}
\def\If{\mytextbf{if}}
\def\Then{\mytextbf{then}}
\def\Else{\mytextbf{else}}
\def\Fi{\mytextbf{fi}}
\def\While{\mytextbf{while}}
\def\Do{\mytextbf{do}}
\def\Od{\mytextbf{od}}

\def\Var#1{\mytextit{#1}}


\def\Class{\mytextbf{class}}
\def\Case{\mytextbf{case}}
\def\Def{\mytextbf{def}}
\def\When{\mytextbf{when}}
\def\Return{\mytextbf{return}}
\def\Nil{\mytextbf{nil}}

%\title{Dynamic Language Embedding Through Incremental Parser Generation}
%\title{Dynamic Syntax Extension Through Incremental Parser Generation}
%\title{A Monolingual Programming Environment with User-Defined
%Syntax}
\title{Towards Dynamically Extensible Syntax}
\author{Tijs van der Storm}
\institute{Centrum Wiskunde \&\ Informatica\\
  Science Park 123\\
  1098 XG Amsterdam}

\begin{document}
\maketitle


\begin{abstract}
  Domain specific language embedding requires either a very flexible
  host language to approximate the desired level of abstraction, -- or
  elaborate tool support for ``compiling away'' embedded notation. The
  former confines the language designer to reinterpreting a given
  syntax. The latter prohibits runtime analysis, interpretation and
  transformation of the embedded language. I tentatively present
  \cherrylisp: a Lisp dialect with dynamically user-definable syntax
  that suffers from neither of these drawbacks.
\end{abstract}

\begin{quote}
  \textit{Jan Heering often speaks fondly of Lisp and much of his
    research has been dedicated to programming environments, syntax,
    semantics, and domain specific languages. On the occasion of his
    retirement I would like to honour him and his work with this
    extended abstract which touches upon some of these subjects.}
\end{quote}

\section{Introduction}
\label{SECT:introduction}

\hfill\begin{minipage}[t]{0.66\linewidth}
\begin{flushright}\small
  \textit{The project of defining M-expressions precisely and
    compiling them or at least translating them into S-expressions was
    neither finalized nor explicitly abandoned. It just receded into
    the indefinite future, ...}\\[0pt]
  \rule{1cm}{0pt} \hfill John McCarthy~\cite{HistoryOfLISP}.
\end{flushright}
\end{minipage}\bigskip

\noindent Domain specific languages (DSLs) are a powerful tool to
increase the level of abstraction in
programming~\cite{WhenAndHowDSL}. Using a notation closer to the
problem domain increases productivity, improves communication with
stakeholders, reduces code size and allows for domain specific
analysis and optimization. Nevertheless, developing a DSL requires a
considerable investment up-front. 

A more approach to DSL engineering is embedding domain specific
notation in a (general purpose) host language. There are basically two
approaches: ``bending'' the syntax of the host language to reach the
desired level of abstraction, or compile-time preprocessing to
assimilate the domain specific notation into the host language by
transformation.  A drawback of the former is that a DSL designer is
confined to the syntactic (and semantic) freedom provided by the host
language. The latter suffers from the fact that domain specific
notation is ``compiled away'': the embedded notation is inaccessible
at runtime, which prohibits dynamic interpretation, analysis,
debugging, pretty printing, optimization and disambiguation of
embedded DSLs. So we find ourselves in a quandary: restricted syntax
with runtime access, or rich syntax without.

Can we combine the best of both embedding worlds? This extended
abstract presents some evidence that the answer is affirmative. I
present the prototype implementation of a monolingual programming
environment~\cite{Monolingual,Eentalig}, called \cherrylisp, which
features arbitrary dynamically user-defined syntax. Embedded syntax is
accessible at runtime, so that language related tooling --
interpreters, typecheckers, debuggers, formatters, optimizers
etc.~\cite{ToolOriented} -- can be developed from within. Key features
of \cherrylisp include:
\begin{enumerate}
\item \textbf{Dynamic syntax extension} the syntax of \cherrylisp can
  be extended at runtime. The abstract syntax trees (ASTs) that are
  the result of parsing embedded syntax are available at runtime for
  analysis, assimilation, translation, interpretation etc.
\item \textbf{Immediate extension} syntax extensions can be immediately used
  in the same file that contains the very expressions to extend the
  language. \cherrylisp's grammar can be modified on the fly during a
  single parse.
\item \textbf{Arbitrary context-free grammars} \cherrylisp's parser is
  based on generalized parsing techniques to allow the use of
  arbitrary context-free grammars for embedded notation. Because of 1,
  should ambiguity arise, it can be dealt with from within \cherrylisp.
\end{enumerate}


\subsubsection*{Related work}
There is ample related work on syntax extension. Almost all related
work employs some form of static assimilation of embedded
notation. Below I briefly discuss some relevant references.

The Lithe language features completely user defined
syntax~\cite{Lithe}. Inspired by Smalltalk-72 it combines a
object-oriented class model with syntax directed translation (SDT) to
attach actions to grammar rules (``methods''). A (parsed) string thus
leads to a sequence of rule applications.  It is unclear which grammar
class is allowed, but ambiguities are explicitly disallowed. ASTs are
not first-class and non-existent at runtime.

Cardelli et al.\cite{ExtensibleSyntax} presented incremental syntax
extensions which respect the scoping rules and type system of the host
language. The syntax extensions are compiled away to the host
language. Their implementation is based on an LL(1) parser so there
are restrictions as to what kind of extensions are possible. An
approach that does not suffer from such restrictions is described
in~\cite{ConcreteSyntaxForObjects}. This work is based on the syntax
definition formalism \textsc{Sdf} ~\cite{SDFManual,EelcoVisserThesis}
which allows arbitrary context-free grammars. Embedded languages are
assimilated into the host language in a compile-time preprocessing
phase. Since \textsc{Sdf} supports arbitrary context-free grammars,
there is a risk of ambiguity; ambiguities must be resolved before
assimilation.

The MS$^2$ programmable syntax macro
system~\cite{ProgrammableSyntaxMacros} provides the macro writer with
full C to define their semantics. Macro expansion has to be completed
before running the program: ``none of it exists at
runtime''. Metamorphic syntax macros are another powerful tool to
extend a host language syntax~\cite{MetaMorphic}. However, ``a syntax
macro must ultimately produce host syntax and thus cannot return user
defined ASTs'' hence the ASTs are not available at runtime.  Finally,
OMeta~\cite{OMeta} is fully dynamic language extension tool, allowing
lexically scoped syntax extensions. It is based on Packrat
parsing. This has the advantage that no ambiguities are possible, but
this also means it does not support arbitrary context-free
grammars. For instance, no left-recursive rules can be used (see
however, \cite{PackratLeftRec}). Additionally, there is no access to
parse-trees, since it is recognition based (in essence similar to
Lithe and inspired by Meta II\cite{MetaII}).

In fact Common Lisp's~\cite{CommonLispTL2} reader macros provide
almost the functionality we are looking for: by hooking into the
reader programmers can seize the opportunity of parsing any embedded
syntax (see~\cite{XExpressions} for an example). \cherrylisp differs
from reader macros in that it uses context-free grammars to define
syntax and a dedicated parsing algorithm producing uniform ASTs. In
other words \cherrylisp provides \textit{declarative} reader
macros. The result can be interpreted, translated, analyzed etc. using
powerful macro facilities just like in Common Lisp~ \cite{MacrosVSBlocks}. 

%As an example of
%an analysis that would be very hard to implement using static syntax
%macros, \cite{MacrosVSBlocks} describes a macro in Common Lisp that
%finds the roots of a Lisp function. It would be very hard to encode
%such an analysis in the reduction semantics of a syntax macro.

\section{Dynamic Syntax Extension in CherryLisp}
\label{SECT:cherrylisp}

%\subsection{Introduction}

\cherrylisp is a Scheme dialect featuring runtime, grammar-based
syntax extension. Below I illustrate a single special form used for
syntax extension.  Cursory knowledge of Lisp is assumed.

%\subsection{Dynamic Syntax Extension}

To allow arbitrary user-defined syntax, \cherrylisp features the
following special form:
\begin{quote}
\texttt{(extend-syntax $[$($l$ $s$ ($x_1 ... x_n$))$]^{*}$)} 
\end{quote}
This special form is used for dynamically extending the current
grammar with context-free productions $s$ ::= $x_1 ... x_n$. The
syntax extension will be immediately available. Resulting AST nodes
will be labeled $l$.  To illustrate how this special form works, let's
extend the syntax with the notation for absolute values $|x|$ by
entering the following \texttt{extend-syntax} invocation in the
read-eval-print-loop (REPL\footnote{The \texttt{>} indicates the
  prompt, the last line is the evaluation result.}):
\begin{quote}
\begin{verbatim}
> (extend-syntax (abs Form ("|" Form "|")))
()
\end{verbatim}
\end{quote}
Since $|x|$ should be a normal expression, we extend the non-terminal
for \cherrylisp expressions (\textit{Form}). After evaluation,
the notation is immediately available:
\begin{quote}
\begin{verbatim}
> |1|
Unbound variable: abs
\end{verbatim}
\end{quote}
A snippet of syntax like this will be converted to internal
S-expressions by the parser. After parsing, the REPL attempts to
evaluate this expression. Since the AST label for absolute value notation is
``abs'' the evaluator attempts to call this function which results in
an unbound variable error. To see what the parser returns, we have to
quote the snippet:
\begin{quote}
\begin{verbatim}
> '|x|
(abs 1)
\end{verbatim}
\end{quote}
To give semantics to this construct, ordinary functions or macros can
be used:
\begin{quote}
\begin{verbatim}
> (define (abs x) (if (>= x 0) x (- 0 x)))
abs
> |1|
1
> |-1|
1
\end{verbatim}
\end{quote}
In this a case a simple function suffices, but the real power lies in
arbitrary combinations of functions and macros. For instance, to give
a more involved example, let's try and self-apply syntax extension by
adding syntax for context-free productions so that grammars can be
written in more natural form:
\begin{quote}
\begin{verbatim}
> (extend-syntax
    (sort Element (Symbol))
    (lit  Element (String))
    (star Element (Symbol "*"))
    (plus Element (Symbol "+"))
    (prod Form (Symbol ":" Symbol "::=" (star Element) ";")))
()
\end{verbatim}
\end{quote}
Note the use of ordinary S-expressions for complex grammar symbols in
last production (\texttt{(star Element)}).  Evaluating this form at
the command-line will add context-free productions defining the syntax
of context-free productions itself to the current grammar of
\cherrylisp. We now can write productions as follows:
\begin{quote}
\begin{verbatim}
> 'abs:Form ::= "|" Form "|";
(prod "abs" "Form" ((sort "Form")))
\end{verbatim}
\end{quote}
This new syntax makes it a lot more easy to define extension grammars.
It is not readily possible to give semantics to these productions
using just functions because of eager argument evaluation. For
example, we could use a macro to interpret \texttt{prod} AST forms, as
follows:
\begin{quote}
%> (define (sort str) (str2sym str))
%sort
\begin{verbatim}
> (define prod (macro (label sym elts)  
    `'(,(str2sym label) ,(str2sym sym) ,elts)))
prod
> abs:Form ::= "|" Form "|";
(abs Form ((sort "Form")))
\end{verbatim}
\end{quote}
Macros in \cherrylisp are similar to ordinary \texttt{lambda}s but
with unevaluated arguments. The quasi-quotation (\texttt{`}) returns
its argument unevaluated except where anti-quotes (\texttt{,}) are
encountered~\cite{QuasiQuotation}. So this macro returns a list with
the label and symbol strings converted to symbols. Note how the result
of entering a syntax production is evaluated on the fly via the
\texttt{prod} macro, returning an S-expression conforming to the rule
format accepted by \texttt{extend-syntax} (with exception of the , as
of yet, unevaluated \texttt{(sort "Form")} construct). In the
following I will assume there is a macro \texttt{syntax} that
completely interprets production ASTs and converts them
\texttt{extend-syntax} invocations. So the following expression
\texttt{(syntax tuple:Form ::= "|" Form "|";)} will effectively add
absolute value notation to the language.

\section{Embedding a Programming Language: Pico}
\label{SECT:pico}

\subsection{Syntax}

\pico~\cite{PicoReport} is a small, \textsc{While}-like language,
featuring assignment, if-then-else, while-do and skip
statements. Supported expressions are literals (strings, naturals),
variables, addition, subtraction and string concatenation. Using the
\texttt{syntax} macro from Section~\ref{SECT:cherrylisp} the syntax of
\cherrylisp is extended with the syntax of \pico (an excerpt of which
is listed in Figure~\ref{FIG:pico-syntax}).

\begin{figure}
\begin{center}
\begin{boxedminipage}[t]{0.9\linewidth}
\small
\begin{verbatim}
program:Program ::= "begin" Declare  Series "end";
declare:Declare ::= "declare" Decls ";";
decl:Decls      ::= Decl;
decls:Decls     ::= Decl ";" Decls;
idtype:Decl     ::= Symbol ":" Type;
assign:Stat     ::= Symbol ":=" Exp;
skip:Stat       ::= "skip";
if:Stat         ::= "if" Exp "then" Series "else" Series "fi";
while:Stat      ::= "while" Exp "do" Series "od";
add:Exp         ::= Exp "+" Term;
sub:Exp         ::= Exp "-" Term;
term:Exp        ::= Term;
cat:Term        ::= Term "||" Factor;
fact:Term       ::= Factor;
bracket:Factor  ::= "(" Exp ")";
\end{verbatim}
\end{boxedminipage}
\end{center}
\caption{Excerpt of a (context-free) grammar for \pico\label{FIG:pico-syntax}}
\end{figure}

To be able to embed \pico programs in \cherrylisp programs, we have to
inject \pico sorts into the primary sort of \cherrylisp. This is where
the \texttt{form} macro comes in. For instance, evaluating
\texttt{(form Stat)} adds the production \textit{Form} ::= ``$[$''
\textit{Stat} ``]'' to the syntax. Recall that \textit{Form} is the
non-terminal capturing \cherrylisp expressions. This means that \pico
statements enclosed in square brackets are valid \cherrylisp
expressions:
\begin{quote}
\begin{alltt}
> '[x := 3]
(form (stat (assign "x" (term (factor (int "3"))))))
\end{alltt}
\end{quote}
The use of the \texttt{form} macro only allows ground \pico terms to
be embedded in \cherrylisp, however, to allow for patterns with holes,
another syntax extension is required. One could, for instance, define
a macro \texttt{(var \textrm{\textit{str}} $S$)} for declaring pattern
variables, which, when evaluated, extends the grammar with the
following productions\footnote{The latter two are actually lexical
  rules that do not contribute to the AST; for the sake of exposition,
  however, I choose to gloss over this detail.}:
\begin{quote}
\begin{alltt}
var:\(S\)     ::= \(S\)-Var;
var:\(S\)-Var ::= ":" \mytextit{str};
var:\(S\)-Var ::= ":" \mytextit{str} Number;
\end{alltt}
\end{quote}
These productions allow \pico syntax patterns to contain
variables. For example, after having evaluated \texttt{(var "exp"
  Exp)}, we can evaluate:
\begin{quote}
\begin{alltt}
> '[x := \Var{:exp}]
(form (stat (assign "x" (var ":exp"))))
\end{alltt}
\end{quote}
Of course, such pattern variables are of little use if these patterns
cannot be used to match \pico source code. Since embedded syntax is
converted to ASTs by the parser, matching can be programmed in plain
\cherrylisp. Such a match function is illustrated as follows:
\begin{quote}
\begin{alltt}
> (match '[x := x + 1] '[x := \Var{:exp}])
((":exp" add (term (factor (id "x"))) (factor (int "1"))))
\end{alltt}
\end{quote}
The \texttt{match} function traverses the structure of both its
arguments, and compares each pair of subtrees it visits. If a variable
is encountered in the pattern, the corresponding subtree of the first
argument is bound to it in an environment structure.  The result of
matching a term against a pattern is the set of bindings created
during matching, in this case the binding ``\textit{:exp}'' to the AST
for ``$x + 1$''. Note that since patterns and pattern variables are
defined using ordinary macros, nothing keeps the user from using
different, embedded-language specific quotes (i.e. other than
``[...]''  and ``:'').

\subsection{Semantics}

Using patterns to embed \pico source code and pattern variables in
matching \pico terms against such patterns allows us to define an
interpreter for \pico in a style similar to the style employed
in~\cite{BHK89,PicoReport}. A function to evaluate \pico statements
is shown below:
\begin{quote}
\begin{alltt}
(define (ev-stat stat env)
  (switch stat
    ([\Skip] env)
    ([\Var{:sym} := \Var{:exp}] (bind sym (ev-exp exp env) env))
    ([\If \Var{:exp} \Then \Var{:series1} \Else \Var{:series2} \Fi]
     (if (ev-exp exp env) 
        (ev-series series1 env)
        (ev-series series2 env)))
    ([\While \Var{:exp} \Do \Var{:series} \Od]
     (if (ev-exp exp env) 
       (ev-stat stat (ev-series series env))
       env)))) 
\end{alltt}
\end{quote}
The function \texttt{ev-stat} evaluates a statement; it assumes two
auxiliary functions: \texttt{ev-series} for evaluating sequences of
statements and \texttt{ev-exp} for evaluating expressions. The
top-level expression of the function is a \texttt{switch} construct.
This macro attempts to match sequences of patterns to its first
argument. If a match is found, the captured variables are made into
normal \cherrylisp bindings so that code right of the pattern can
refer to them. For instance, the pattern for assignment contains a
variable \Var{:sym} which matches a \pico variable. In the right-hand
side of this case the captured variable is available as \texttt{sym}.


\section{Implementation: GLR \&\  IPG}

The current version of \cherrylisp is implemented in
Ruby\footnote{http://www.ruby-lang.org}. There are three important
parts: the parser, de parser generator (the ``grammar'') and the
interpreter.  The parser algorithm is the Generalized LR (GLR)
algorithm as described in~\cite{RekersThesis}. The LR(0) parsetable
generation algorithm is both incremental and lazy
(IPG)~\cite{RekersThesis, IncrementalParserGeneration}. I have
instantiated both the parsing and parser generation algorithms in
scannerless style; this obviates the need for a separate lexical
analysis phase\footnote{... but might require some provision to
  enforce longest match for certain lexical sorts, such as
  identifiers; again, I choose to gloss over this technical detail.}.

The basic configuration of the \cherrylisp interpreter proceeds as
follows:
\begin{quote}
\begin{alltt}
\Var{g} = IPGGrammar.new("boot.grammar")
\Var{p} = GLRParser.new(\Var{g})
\Var{e} = Evaluator.new(\Var{g})
\Var{p}.add_reduction_observer(\Var{e})
\end{alltt}
\end{quote}
First the IPG parser generator is initialized with the initial grammar
containing the syntax of the host language, -- in this case a small
syntax for S-expressions suffices. Then both the parser and evaluator
are initialized with the parser generator. The GLR algorithm will use
$g$ to parse text and produce AST. The interpreter has to know about
the same $g$ in order to add productions to it. Finally, following the
Observer design pattern, the interpreter is added as an observer of
the GLR algorithm. In this case it means that the interpreter will be
notified of reductions performed by the GLR algorithm.

Figure~\ref{FIG:eval} illustrates some of the internals of the
interpreter. Evaluators are initialized (through \texttt{initialize})
with a grammar and a fresh environment\footnote{In Ruby, instance
  variables start with an ``@''.}. The next method,
\texttt{notify\_reduction} is called during runs of the GLR parsing
algorithm. The argument is the AST created in a reduce action. If this
AST is of the \texttt{extend-syntax} kind, the grammar is immediately
modified. 

\begin{figure}
\small
\begin{quote}
\begin{minipage}[t]{0.55\linewidth}
\begin{alltt}
\Def initialize(\Var{grammar})
  \Var{@grammar} = \Var{grammar}
  \Var{@env} = Environment.new
\End

\Def notify_reduction(\Var{form})
  \If \Var{node}.first == "extend-syntax"
    \Var{@grammar}.extend(\Var{node}.rest)
  \End
\End
\end{alltt}
\end{minipage}
\begin{minipage}[t]{0.5\linewidth}
\begin{alltt}
\Def eval(\Var{form}, \Var{env})
  ...
  \Case \Var{form}.first
   \When "extend-syntax" 
     \Return \Nil
    ...
  \End
\End
\end{alltt}
\end{minipage}
\end{quote}
\caption{Simplified method skeleton of the \cherrylisp
  interpreter.\label{FIG:eval}}
\end{figure}
Finally, the main content of the interpreter is in the \texttt{eval}
method. It consists of a large \textbf{case} statement that dispatches
on the incoming expression (\textit{form}). For the sake of brevity,
only the cases relevant to syntax extension are shown. If the
interpeter encounters an \texttt{extend-syntax} form it just returns
\textbf{nil}, since the arguments to it have already been evaluated
\textit{during} parsing (in \texttt{notify\_reduction}), and from the
fact that \textit{form} is a parameter to \texttt{eval} it naturally
follows that parsing (with respect to \textit{form}) has already
finished. 

\section{Future Work}

\paragraph{Understanding} First of all, further research should
concentrate at deeper understanding of the consequences of modifying
LR parsetables \textit{during} parsing. The experiments with the
current prototype of \cherrylisp are encouraging (however naive the
implementation may be). Nevertheless, it is unclear whether there are
circumstances where the current scheme would not work. For instance,
currently, there is control over what would happen if, in the case of
ambiguity, the parsetable is modified ``concurrently'' in different,
possibly interacting, ways (!). Similarly, there is no rollback for
undoing parsetable modifications in the case that a certain path
exercised by the GLR algorithm does \textit{not} eventually lead to a
valid parse. In other words, currently, parsetable modifications do
not obey the stack discipline of the GLR algorithm and ignore the its
quasi concurrency. A purely function implementation of GLR using a
persistent datastructure for the parsetable could be a viable approach
to solve these problems.

\paragraph{Bootstrapping} Lisps have a very simple syntax: \cherrylisp
therefore begs to be bootstrapped. In fact, I think that the following
five (context-free) productions are should be enough to bootstrap
\cherrylisp to having a full-fledged Lisp syntax:
\begin{quote}\small
\begin{minipage}[t]{0.6\linewidth}
\begin{alltt}
nil:Form    ::= "(" ")"
pair:Form   ::= "( Prefix "." Form ")"
form:Prefix ::= Form
\end{alltt}
\end{minipage}
\begin{minipage}[t]{0.4\linewidth}
\begin{alltt}
num:Form ::= Number
sym:Form ::= Symbol
\end{alltt}
\end{minipage}
\end{quote}
These rules suffice for Lisp expressions consisting of dotted pairs
(e.g. \texttt{(1 . 2)}), number literals and symbols. Using the syntax
extension mechanism of \cherrylisp any other syntax can be added: list
syntax, string literals, quotation syntax, etc.

\paragraph{Syntax restrictions} Currently, \cherrylisp only allows the
current syntax to be extended, but not restricted. It can be useful
however, to sometimes undo a syntax extension: this will also help
to minimize possible ambiguities arising from the interaction among
embedded languages and the host language. The incremental parser
generation algorithm IPG already caters for the removal of productions
from a grammar.


\paragraph{Lexically scoped syntax extensions} This area of future
work is closely related to the previous point. Similar to the approach
of~\cite{ExtensibleSyntax} one would like to restrict the grammar for
a certain region of source code without having to introduce extra
non-terminals. Lexically scoped syntax extensions can be implemented
by performing extension and restriction using a stack
discipline. However, there are techniques to (temporarily) restrict a
grammar without adding and/or removing
productions~(\cite{RekersThesis}, Chapter 3).

I have performed some experiments based on adding and subsequently
removing productions from a grammar the results of which are somewhat
encouraging. Scoped grammar modification would cater for idioms like:
\begin{quote}
\begin{alltt}
(with-syntax (abs:Form ::= "|" Form "|";) |x - y|)
\end{alltt}
\end{quote}
This would effectively set up a kind of \textit{syntax jail}: the
absolute value notation is only available within the confines of the
\texttt{with-syntax} construct. Additionally, one could imagine syntax
jails where a completely different grammar is active and not just a
certain \textit{extended} grammar (as in the example). Further
research is required however to understand the possible effects of
destructively (i.e. non-conservatively) modifying the parse table
during parsing.

\paragraph{First class grammars} To make the previous construct even
more powerful the current grammar (or really any kind of grammar)
should be a first-class value in the runtime system. Grammars could
then be assigned to variables just like any other value. In
combination with lexically scoped syntax regions, this would allow
idoms like: \texttt{(with-syntax abs-syntax |-1|)}, where the
variable \texttt{abs-syntax} holds the productions defining the
absolute value notation.

Additionally, first class grammars would enable grammars to be
processed within \cherrylisp itself. In combination with Lisp's
powerful macro facilities, one could generate default source code
formatters~\cite{PrettyPrintingSoftwareReengineering} and AST
manipulation programming interfaces~\cite{APIGen}.

\paragraph{First class parse trees} The current focus of \cherrylisp
is syntax extension for DSL embedding. In this case, abstract syntax
is often sufficient for analysis, interpretation and transformation
purposes. However, for reverse- and reengineering purposes concrete
syntax trees are needed. Such trees could readily be made available in
\cherrylisp since the GLR parsing algorithm, in fact, already produces
them\footnote{To be completely correct, it produces shared packed
  parse forests; see \cite{RekersThesis}.}. Concrete syntax trees
preserve information on layout, literals, and comments, -- essential
information, for instance, when implementing automatic refactorings or
source level debugging facilities.


\section{Conclusion}

I have described a simple, yet very powerful approach to dynamic,
incremental and immediate syntax extension. Embedded syntax is
transformed by the parser into S-expressions which are readily
available for analysis, evaluation, expansion or any kind of
processing one could conceive of in a Lisp-like language. The current
prototype is admittedly naive, and further work should focus on deeper
understanding of the dynamics of syntax extension.

{
\bibliographystyle{abbrv}
\bibliography{dynexsyn}
}


\end{document}
